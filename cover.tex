\documentclass[11pt,a4paper,twoside,openright]{report}

\usepackage{graphicx}

\setlength{\oddsidemargin} {2. cm}
\setlength{\evensidemargin} {2. cm}
\addtolength{\oddsidemargin} {-0.4 cm}
\addtolength{\evensidemargin} {-0.4 cm}
\linespread{1.1}


\usepackage{libertine}
\usepackage[T1]{fontenc}
\usepackage[libertine,cmintegrals,cmbraces,vvarbb]{newtxmath}

% or just this: \usepackage[sc]{mathpazo}


\pagestyle{empty}

\begin{document}
\vspace*{-1.5cm} 
\begin{center}
  \large
  POLITECNICO DI MILANO\\
  \normalsize
  
  Scuola di Ingegneria Industriale e dell'Informazione\\
  Corso di Laurea Magistrale in Ingegneria Informatica\\
  
  \vspace*{0.3cm}

  \begin{figure}[htbp]
    \begin{center}
      \includegraphics[width=3.5cm]{./src/logo_pdm.pdf}
    \end{center}
  \end{figure}
  \vspace*{0.3cm} \LARGE



  A METHODOLOGY BASED ON FUNCTIONAL LANGUAGES
           FOR THE DESIGN OF HARDWARE CRYPTOGRAPHIC 
           PRIMITIVES RESISTANT TO SIDE-CHANNEL ATTACKS\\



           \vspace*{.75truecm}
\end{center}
\vspace*{2.5cm} \large
\begin{flushleft}


  {Advisor (Politecnico di Milano)}:\\ Prof. Vittorio ZACCARIA\\
  ~\\
  Co-Advisor (STMicroelectronics):\\ Ing. Filippo MELZANI

\end{flushleft}
\vspace*{1.0cm}
\begin{flushright}

  Tesi di Laurea Magistrale di:\\ Lorenzo DELLEDONNE\\ matricola 804661
  % Master's Thesis of:\\ Lorenzo DELLEDONNE\\ matricola 804661


\end{flushright}
\vspace*{1.0cm}
\begin{center}


  Academic Year 2016-2017 
\end{center} 
\clearpage
\end{document}
