
La sicurezza delle primitive hardware che realizzano algoritmi crittografici è gravemente 
compromessa da una classe di vulnerabilità dovute alla dispersione accidentale, da parte dei 
dispositivi fisici, di informazioni sensibili. 
Gli attacchi contro le implementazioni, conosciuti come attacchi \emph{``side-channel''},
sfruttano fonti di output non intenzionali, % presenti nelle implementazioni fisiche, 
come il \emph{consumo istantaneo di potenza}, per infrangere algoritmi crittografici 
considerati sicuri.
Di conseguenza, le primitive crittografiche hardware devono obbligatoriamente essere 
equipaggiate con particolari \emph{contromisure}, in grado di mitigare le vulnerabilità 
side-channel delle implementazioni fisiche. 

Tuttavia, c'è una evidente carenza di metodologie consolidate per gestire
il design e l'implementazione di primitive crittografiche protette contro attacchi side-channel.
Come conseguenza, attualmente il processo di progettazione di primitive crittografiche protette 
contro gli attacchi side-channel è parecchio oneroso nonché soggetto ad errori. 
L'obiettivo principale di questa tesi è quello di esplorare la fattibilità di una nuova metodologia di 
progettazione hardware, in grado di ridurre il divario % esistente.....
tra le metodologie di progettazione esistenti, da una parte, e la variegata collezione di tecniche 
di crittoanalisi per la valutazione delle vulnerabilità side-channel dei circuiti crittografici, dall'altra.

Proponiamo di fare affidamento su speciali pattern di programmazione funzionale per lo sviluppo 
di una serie di \emph{domain-specific languages} (DSL) come base per la specifica, 
l'implementazione e la verifica di design sicuri.
Abbiamo sviluppato un framework di verifica basato su un nuovo DSL \emph{``polimorfico''}, 
integrato in Haskell (\emph{domain-specific embedded language}, o DSEL), che può essere 
arbitrariamente valutato secondo diverse interpretazioni semantiche.
In questo modo, siamo in grado di eseguire le varie procedure di verifica necessarie, così come 
di generare automaticamente una specifica RTL del circuito (scritta in VHDL o Verilog), 
sintetizzabile per mezzo di strumenti industry-standard, a partire dalla medesima specifica ad 
alto livello.

Forniamo infine una valutazione sperimentale sull'efficacia del nostro approccio nel facilitare 
l'esplorazione dello spazio delle soluzioni, per quanto riguarda le primitive hardware protette da 
contromisure, e ne proveremo la concreta fattibilità.