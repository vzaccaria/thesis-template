\newpage
\chapter*{Sommario}

% \addcontentsline{toc}{chapter}{Sommario}

% Sommario della tesi in italiano.
La sicurezza delle primitive hardware che realizzano algoritmi crittografici è gravemente 
compromessa da una classe di vulnerabilità dovute alla dispersione accidentale, da parte dei 
dispositivi fisici, di informazioni sensibili. 

Gli attacchi contro le implementazioni, conosciuti come attacchi \emph{``side-channel''},
violano il \emph{threat model} tipico dei sistemi crittografici, la cui sicurezza è da sempre stata 
valutata concentrandosi esclusivamente sulle proprietà matematiche degli algoritmi e dei 
protocolli impiegati, sotto l'ipotesi che i dati confidenziali vengano sempre manipolati in modo 
sicuro, senza cioè mai esporre nessuna informazione al di fuori degli input ed output previsti.

Gli attacchi side-channel, al contrario, sfruttando fonti di output non intenzionali presenti nelle 
implementazioni fisiche, come ad esempio il loro \emph{consumo istantaneo di potenza}, sono 
spesso in grado di infrangere algoritmi crittografici considerati sicuri.
In effetti, negli ultimi due decenni è diventato chiaro come gli attacchi side-channel siano non 
solo fattibili, ma addirittura pratici ed efficienti da condurre. Per queste ragioni, rappresentano 
una seria minaccia.

Di conseguenza, le primitive crittografiche hardware devono obbligatoriamente essere 
equipaggiate con specifiche contromisure, in grado di mitigare le vulnerabilità side-channel 
delle implementazioni fisiche. 
L'obiettivo dei progettisti è dunque quello di realizzare dispositivi a prova di manomissione, 
che assicurino un livello di sicurezza adeguata, anche se realizzati da circuiti elettronici che 
inevitabilmente rilasciano informazioni. % ....disperdono
Le contromisure contro attacchi side-channel possono essere progettate e impiegate a 
vari livelli di astrazione, come ad esempio a livello di algoritmo, dei gate logici o persino dei 
transistor. 
In questa tesi % ....Nella nostra discussione
ci concentreremo esclusivamente su \emph{contromisure algoritmiche}, che tipicamente 
operano cercando di ridurre la dipendenza statistica di ciascun ``valore intermedio'', 
ottenuto durante l'elaborazione, %...la computazione
rispetto ai dati sensibili da proteggere.
Esistono diverse strategie per realizzare contromisure algoritmiche. Le più diffuse sono quelle 
basate su metodi di \emph{random masking}, che si servono di una serie di input aggiuntivi 
chiamati ``maschere'', uniformemente distribuiti su un intervallo opportuno, per partizionare 
l'elaborazione dei valori intermedi % ....confidenziali
in vari ``share'' (fisicamente osservabili) i quali, individualmente, risultano statisticamente 
indipendenti dai dati sensibili.

La progettazione e l'implementazione di primitive crittografiche ``provably secure'' su circuiti 
concreti è oggigiorno un'impresa alquanto problematica. 
Il processo di verifica della sicurezza delle contromisure è purtroppo ancora un compito oneroso 
e prevalentemente manuale.
In effetti, la capacità di scoprire prontamente le vulnerabilità side-channel 
durante in ciclo di sviluppo % ....sin da subito {early in the design cycle}
è cruciale per l'efficienza della produzione di primitive protette.
Tuttavia, c'è una evidente carenza di metodologie consolidate per affrontare questo genere di 
problematiche durante la fase di progettazione.
Gli strumenti attualmente in uso sono collegati in modo non ottimale all'interno del flusso di 
lavoro e, per di più, originalmente non erano nemmeno stati pensati per il compito di analizzare 
le vulnerabilità side-channel. Come conseguenza, attualmente il processo di progettazione di 
primitive crittografiche protette contro gli attacchi side-channel è parecchio oneroso nonché 
soggetto ad errori. 
Pertanto, c'è una forte necessità di strumenti automatizzati ed efficienti per la valutazione e la 
verifica, a \emph{design-time}, della sicurezza delle contromisure impiegate.

L'obiettivo principale di questa tesi è quello di esplorare la fattibilità di una nuova metodologia di 
progettazione hardware che affronti precisamente queste problematiche, consentendo il rapido 
sviluppo di primitive crittografiche hardware robuste e sicure, 
proteggendo i progettisti dagli errori. % ....
%
Ciò che ci interessa maggiormente è dunque chiudere il divario esistente tra le metodologie di 
progettazione hardware, da una parte, e la variegata collezione di tecniche di crittoanalisi per la 
valutazione delle vulnerabilità dei circuiti crittografici, dall'altra.

A questo scopo, proponiamo di fare affidamento su determinati pattern di programmazione 
funzionale per lo sviluppo di una serie di \emph{domain-specific languages} (DSL) come base 
per la specifica, l'implementazione e la verifica di design sicuri.
%
% Il fondamento del nostro approccio consiste nel....
Il nostro approccio si fonda nel descrivere i circuiti per mezzo di specifiche ad alto livello, scritte 
con il linguaggio funzionale Haskell. Ciò significa che % Vale a dire che....
% ....sono rappresentate a tutti gli effetti da programmi software.
le specifiche circuitali sono a tutti gli effetti rappresentazioni software eseguibili.
Un tale livello di astrazione % ... più elevato % .....
ci consente di inserire, nel codice delle specifiche, la strumentazione necessaria per realizzare le 
procedure di verifica funzionale e altresì di valutazione delle proprietà non-funzionali relative a 
vulnerabilità side-channel.
%
% Pertanto, % .......along these lines
% Seguendo questo approccio, 
Seguendo questo modello, 
abbiamo sviluppato un framework di verifica basato su un nuovo DSL \emph{``polimorfico''}, 
integrato in Haskell (\emph{domain-specific embedded language}, o DSEL), che può essere 
arbitrariamente valutato secondo diverse interpretazioni semantiche.
In particolare, una speciale interpretazione garantisce che le specifiche circuitali ad alto livello 
siano sempre tradotte correttamente in descrizioni RTL industry-standard, scritte ad esempio in 
VHDL o Verilog.
In questo modo, siamo in grado di eseguire % la varietà di ....
le varie procedure di verifica a partire dal medesimo documento di specifica circuitale che verrà 
poi usato come base per la sintesi logica, mentre % .....
il codice di strumentazione rimane sempre \emph{trasparente} al processo di sintesi di alto 
livello.

Infine, dimostreremo sperimentalmente l'efficacia del nostro approccio %nell'assistere 
nel facilitare l'esplorazione dello spazio delle soluzioni, per quanto riguarda le primitive hardware 
protette contro attacchi side-channel, e ne proveremo la concreta fattibilità.
% in uno scenario reale .....
% in un contesto industriale.