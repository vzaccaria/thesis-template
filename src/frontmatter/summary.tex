\newpage
\chapter*{Abstract}

% \addcontentsline{toc}{chapter}{Abstract}

% This thesis points out the value of using a modern functional language for
% designing industrial-grade cryptographic hardware primitives resistant against
% side-channel attacks.

% The security of hardware cryptographic primitives can be broken by \emph{``side-channel''} attacks

The security of hardware cryptographic primitives is critically affected 
by a class of vulnerabilities due to the unintended leakage, occurring in physical 
devices, of sensitive information. 
Attacks against implementations, known as \emph{``side-channel''} attacks, exploit 
unexpected physical output channels, such as the \emph{instantaneous 
power consumption}, to break cryptographic algorithms otherwise thought as secure.
Consequently, hardware cryptographic primitives must be equipped with specific 
\emph{countermeasures} that mitigate the side-channel vulnerability of the physical 
implementations. 

However, there is an obvious lack of established methodologies for dealing with
the design and the implementation of cryptographic primitives protected against side-channel 
attacks. As a result, the current development process is both onerous and error-prone.
The main goal of this thesis is to explore the feasibility of a new hardware design methodology,
suitable for closing the gap that exists between current hardware design practices and the 
diverse set of side-channel cryptanalysis techniques for the assessment of the security 
vulnerability of cryptographic circuits.

We propose to rely on special functional programming patterns for developing a 
number of \emph{domain-specific languages} (DSLs) as the basis for the specification, the 
implementation and the verification of secure designs.
We built an instrumentation framework that relies on a new \emph{``polymorphic'' 
domain-specific embedded language} (DSEL), within Haskell,
which can be arbitrarily re-evaluated according 
to multiple language interpretations. In this way, we manage to perform the variety of 
validation tasks needed, as well as to generate an industry-standard synthesizable RTL description (written in VHDL or Verilog), % by using the
starting from the very same high-level circuit specification.

We provide an experimental evaluation about the effectiveness of our approach in assisting 
the design-space exploration of protected primitives, and we demonstrate its actual feasibility in a real-world scenario.