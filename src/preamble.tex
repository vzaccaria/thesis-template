% !TEX root = thesis.tex


% ===== XeLaTeX I/O setup ===== %
\usepackage[english]{babel}
%\usepackage{polyglossia}
%\setmainlanguage{english}
% \usepackage{mathspec} % loads fontspec as well
% \defaultfontfeatures{Ligatures=TeX}
% \usepackage{lmodern}

\usepackage{libertine}
\usepackage[utf8]{inputenc}
\usepackage[T1]{fontenc}
\usepackage{amsmath,amssymb,amsthm}
\usepackage[libertine,cmintegrals,cmbraces,vvarbb]{newtxmath}

% ===== Page Layout ===== %
\usepackage{fancyhdr}
\usepackage{etoolbox}
% \setlength{\paperwidth}{16cm}
% \setlength{\paperheight}{24cm}
\setlength{\oddsidemargin} {2. cm}
\setlength{\evensidemargin} {2. cm}
\addtolength{\oddsidemargin} {-0.4 cm}
\addtolength{\evensidemargin} {-0.4 cm}
\linespread{1.1}


\usepackage{acronym}
\usepackage{afterpage}
\usepackage{enumitem}
\usepackage{float}
\usepackage{graphicx}
\usepackage{needspace}
\usepackage[numbers]{natbib}
\usepackage{rotating}
\usepackage[font=small, labelfont=sc, textfont=it]{caption}
\usepackage[position=b, labelfont=rm]{subcaption}
\usepackage{tabularx}
\usepackage{url}
\usepackage{lipsum}
\usepackage{xargs} 
\usepackage[colorinlistoftodos,prependcaption,textsize=tiny]{todonotes}
\newcommandx{\unsure}[2][1=]{\todo[linecolor=red,backgroundcolor=red!25,bordercolor=red,#1]{(VZ): non sono sicuro di questo}}
\newcommandx{\change}[2][1=]{\todo[linecolor=blue,backgroundcolor=blue!25,bordercolor=blue,#1]{(VZ): cambiare in "#2"}}
\newcommandx{\info}[2][1=]{\todo[linecolor=red,backgroundcolor=red!25,bordercolor=red,#1]{(VZ): #2}}
\newcommandx{\thiswillnotshow}[2][1=]{\todo[disable,#1]{(VZ): #2}}

% \usepackage{scrextend}
\usepackage{framed}
\usepackage{fancyvrb}
\DeclareTextCommand{\nobreakspace}{T1}{\leavevmode\nobreak\ }


% ===== Fonts ===== %



% ===== Styling ===== %
\captionsetup{labelsep=period}
\captionsetup[table]{skip=1.5pt}
% \DeclareCaptionLabelSeparator{dot-space}{.~~}
% \captionsetup{labelsep=dot-space}

\setlist[description]{leftmargin=\parindent,labelindent=\parindent}

\urlstyle{same}

\usepackage{tabstackengine}
\stackMath
% === %


% ===== Footnotes ===== %
\usepackage{scrextend}
\makeatletter
\def\@xfootnote[#1]{%
  \protected@xdef\@thefnmark{#1}%
  \@footnotemark\@footnotetext}
\makeatother
\deffootnote{0em}{1.6em}{\thefootnotemark.\enskip}

% === %

% ===== Definitions and Theorems ===== %
\theoremstyle{definition}
\newtheorem{definition}{Definition}
\newtheorem{assumption}{Assumption}
\newtheorem{corollary}{Corollary}
\newtheorem{remark}{Remark}
\newtheorem{example}{Example}

% \linespread{0.99}
% \theoremstyle{plain}
% \newtheorem{theorem}{Theorem}
% === %


% ===== Algorithms and code ===== %
\usepackage{listings}
\lstset{
  basicstyle        = \small\ttfamily,
  breaklines        = true,
  breakatwhitespace = true,
  keywordstyle      = \bfseries,
  captionpos        = t,
  commentstyle      = \ttfamily,
  commentstyle      = \ttfamily\itshape,
  % commentstyle    = \small\rmfamily\itshape,
  columns           = fullflexible,
  frame             = lines,
  %language         = Haskell,
  %xleftmargin      = .1\textwidth, xrightmargin=.1\textwidth,
  keepspaces        = true
}

% \usepackage{minted}

\usepackage{algorithm}
\usepackage{algorithmicx}
\usepackage{algpseudocode}
\makeatletter
\def\BState{\State\hskip-\ALG@thistlm}
\makeatother
% === %


% ===== Hyphenation ==== %
\hyphenation{a-na-ly-sis}
\hyphenation{ab-s-trac-ting}
\hyphenation{au-to-ma-ti-cal-ly}
\hyphenation{au-to-ma-tiz-za-ti}
\hyphenation{chiu-de-re}
\hyphenation{cor-re-spon-ding}
\hyphenation{crypt-a-na-ly-sis}
\hyphenation{de-cla-ra-tion}
\hyphenation{en-for-cing}
\hyphenation{e-va-lu-a-ting}
\hyphenation{e-va-lu-a-tion}
\hyphenation{e-va-lu-a-tions}
\hyphenation{ex-pli-cit-ly}
\hyphenation{Ha-skell}
\hyphenation{im-pie-ga-te}
\hyphenation{in-du-stri-a-li-za-tion}
\hyphenation{in-stan-tia-ted}
\hyphenation{in-te-re-s-ting}
\hyphenation{me-tho-do-lo-gy}
\hyphenation{mo-du-lar}
\hyphenation{na-tu-ral}
\hyphenation{o-pe-ra-tion}
\hyphenation{o-ri-gi-nal}
\hyphenation{phy-si-cal}
\hyphenation{po-pu-lar}
\hyphenation{pri-mi-ti-ve}
\hyphenation{pro-blems}
\hyphenation{pro-du-ct}
\hyphenation{pro-pa-ga-tion}
\hyphenation{pro-per-ly}
\hyphenation{pro-per-ty}
\hyphenation{Quick-Check}
\hyphenation{ran-do-mi-zed}
\hyphenation{re-gu-lar}
\hyphenation{see-min-gly}
\hyphenation{si-mu-la-ted}
\hyphenation{si-mu-la-tion}
\hyphenation{spe-ci-fi-ca-tion}
\hyphenation{spe-ci-fi-ca-tio-ns}
\hyphenation{spe-ci-fi-ed}
\hyphenation{va-li-da-tion}
\hyphenation{va-lu-es}
\hyphenation{ve-ri-fi-ca-tion}
\hyphenation{Ve-ri-log}
\hyphenation{vio-la-no}

% ===== Custom commands ==== %
\newcommand{\CLaSH}{CλaSH}
%\newcommand{\CLaSH}{C$\lambda$aSH}
\newcommand{\dollar}{\$}
\newcommand{\ampersand}{\&}
\newsavebox{\LstFootBox}
\newsavebox{\LstFootBoxB}


%\usepackage[bottom]{footmisc}
%\usepackage[activate={true,nocompatibility},final,tracking=true,kerning=true,spacing=true,factor=1100,stretch=10,shrink=10]{microtype} % see http://www.khirevich.com/latex/microtype/
% \UseMicrotypeSet[protrusion]{basicmath} % disable protrusion for tt fonts
%\usepackage{subfigure}
%\usepackage{upquote}
\usepackage{pdfpages}


% ===== Pandoc commands ===== %
\setlength{\emergencystretch}{3em}  % prevent overfull lines
\providecommand{\tightlist}{%
  \setlength{\itemsep}{0pt}\setlength{\parskip}{0pt}}

% set default figure placement to htbp
\makeatletter
\def\fps@figure{htbp}
\makeatother

\lstset{aboveskip=10pt,belowskip=20pt}
